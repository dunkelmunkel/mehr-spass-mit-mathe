\documentclass{../cssheet}

%--------------------------------------------------------------------------------------------------------------
% Basic meta data
%--------------------------------------------------------------------------------------------------------------

\title{Mathe-Angst}
\author{Prof. Dr. Christian Spannagel}
\date{\today}
\hypersetup{%
    pdfauthor={\theauthor},%
    pdftitle={\thetitle},%
    pdfsubject={Aufgaben im Kontext der Veranstaltung Mehr Spaß mit Mathe!},%
    pdfkeywords={algebra}
}

%--------------------------------------------------------------------------------------------------------------
% document
%--------------------------------------------------------------------------------------------------------------

\begin{document}
\printtitle

Die Basis ist der Text von Szucs, D. \& Toffalini, E. (2023). \emph{Maths anxiety and subjectiveperception of control, valueand success expectancy inmathematics}. \emph{Royal Society Open Science}, 10, 231000.

\section*{Phase~1: Individuelle Bearbeitung}

\textbf{Impuls 1:} Schreibe fünf Erkenntnisse der Studie auf, die du für am wichtigsten hältst.

\textbf{Impuls 2:} Welche Ergebnisse der Studie hast du erwartet, welche haben dich überrascht, und warum?

\textbf{Impuls 3:} Bei dem Text handelt es sich um eine richtige Studie mit einigen technischen Details. Wir müssen nicht alles im Detail verstehen, aber vielleicht hast du Fragen, die dich interessieren? Schreibe alle Fragen zum Text auf, die du gerne klären würdest.

\section*{Phase~2: Austausch in eurer Lerngruppe}

\textbf{Impuls 4:} Sprecht über eure individuellen Ergebnisse aus den ersten drei Impulsen und führt eure Überlegungen zusammen.

\vspace*{10mm}
\printlicense

\printsocials


\end{document}
