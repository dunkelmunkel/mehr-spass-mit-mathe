\documentclass{../cssheet}

%--------------------------------------------------------------------------------------------------------------
% Basic meta data
%--------------------------------------------------------------------------------------------------------------

\title{Motivation}
\author{Prof. Dr. Christian Spannagel}
\date{\today}
\hypersetup{%
    pdfauthor={\theauthor},%
    pdftitle={\thetitle},%
    pdfsubject={Aufgaben im Kontext der Veranstaltung Mehr Spaß mit Mathe!},%
    pdfkeywords={algebra}
}

%--------------------------------------------------------------------------------------------------------------
% document
%--------------------------------------------------------------------------------------------------------------

\begin{document}
\printtitle

\textbf{Vorbemerkung:}  Diese Aufgaben teilen sich in zwei Phasen: In Phase~1 reflektiert ihr eure eigene Motivation im Mathe-Unterricht und in Mathe-Lehrveranstaltungen. In Phase~2 trefft ihr euch in eurer Lerngruppe und tauscht eure Gedanken aus.

\section*{Phase~1: Reflexion}

\textbf{Aufgabe 1 (Was ist Abstraktion?):}  Was ist Abstraktion für euch? Wie würdet ihr jemandem erklären, was Abstraktion ist? Habt ihr ein anschauliches Beispiel für Abstraktion, z.\,B. aus dem Alltag?

\textbf{Aufgabe 2 (Wozu Abstraktion?):} Wozu ist Abstraktion gut?


\section*{Phase~2: Austausch in der Lerngruppe}

\textbf{Aufgabe 3 (Wissenschaft Mathematik):} Die Mathematik ist die Wissenschaft der Muster und Strukturen. Was hat das mit Abstraktion zu tun?



\printlicense

\printsocials


\end{document}
