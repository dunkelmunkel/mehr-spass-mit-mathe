\documentclass{../cssheet}

%--------------------------------------------------------------------------------------------------------------
% Basic meta data
%--------------------------------------------------------------------------------------------------------------

\title{Motivation}
\author{Prof. Dr. Christian Spannagel}
\date{\today}
\hypersetup{%
    pdfauthor={\theauthor},%
    pdftitle={\thetitle},%
    pdfsubject={Aufgaben im Kontext der Veranstaltung Mehr Spaß mit Mathe!},%
    pdfkeywords={algebra}
}

%--------------------------------------------------------------------------------------------------------------
% document
%--------------------------------------------------------------------------------------------------------------

\begin{document}
\printtitle

\textbf{Vorbemerkung:}  Diese Aufgaben teilen sich in zwei Phasen: In Phase~1 reflektiert ihr eure eigene Motivation im Mathe-Unterricht und in Mathe-Lehrveranstaltungen. In Phase~2 trefft ihr euch in eurer Lerngruppe und tauscht eure Gedanken aus. Haltet eure Ergebnisse übersichtlich fest, zum Beispiel in einer Mindmap!

Die Basis aller Überlegungen ist der Text von Deci und Ryan (1993).\footnote{Deci, E. L. \& Ryan, R. M. (1993) Die Selbstbestimmungstheorie der Motivation und ihre Bedeutung für die Pädagogik. \emph{Zeitschrift für Pädagogik, 39}, 2, 223--238.}

\section*{Phase~1: Reflexion}

\textbf{Aufgabe 1 (Formen der Motivation):}  Welche Formen der Motivation kennst du aus deiner \glqq{}Mathe-Lern-Vergangenheit\grqq{} an der Schule und der Hochschule? Warst du eher extrinsisch oder intrinsisch motiviert? Und falls du eher extrinsisch motiviert warst, um welche Form der extrinsischen Verhaltensregulation hat es ich gehandelt?

\textbf{Aufgabe 2 (Veränderungen):} Wie hat sich deine Motivation beim Mathematiklernen im Laufe deiner Schulzeit und deines Studiums verändert? Woran lag das?

\textbf{Aufgabe 3 (Bedürfnisse):}  Welche Rolle spielen für dich die Bedüfrnisse nach Autonomie, Kompetenzerleben und sozialer Eingebundenheit im Kontext von Mathematiklernen? Welche Lehrer:innen oder Dozent:innen haben diese Bedürfnisse bedient, und wie?



\section*{Phase~2: Austausch in eurer Lerngruppe}

\textbf{Aufgabe 4 (Gemeinsamkeiten und Unterschiede):} Tauscht euch in eurer Lerngruppe über eure persönlichen Reflexionen aus. Könnt ihr Gemeinsamkeiten finden? Worin unterscheiden sich eure Wahrnehmungen? Welche neuen Perspektiven ergeben sich durch eure Diskussion?


\vspace*{10mm}
\printlicense

\printsocials


\end{document}
