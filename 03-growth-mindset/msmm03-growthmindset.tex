\documentclass{../cssheet}

%--------------------------------------------------------------------------------------------------------------
% Basic meta data
%--------------------------------------------------------------------------------------------------------------

\title{Growth Mindset}
\author{Prof. Dr. Christian Spannagel}
\date{\today}
\hypersetup{%
    pdfauthor={\theauthor},%
    pdftitle={\thetitle},%
    pdfsubject={Aufgaben im Kontext der Veranstaltung Mehr Spaß mit Mathe!},%
    pdfkeywords={algebra}
}

%--------------------------------------------------------------------------------------------------------------
% document
%--------------------------------------------------------------------------------------------------------------

\begin{document}
\printtitle

\textbf{Vorbemerkung:}  Der erste wissenschaftliche Text in dieser Veranstaltung auf Englisch — für viele von euch sicher eine ungewohnte und herausfordernde Erfahrung. Perfekt, um genau das zu tun, was im Artikel Thema ist: über euer eigenes Mindset nachdenken!

Die Basis ist der Text von Dweck, C. (2008). \emph{Mindsets and  Math/Science Achievement}.

\section*{Phase~1: Reflexion}

\textbf{Impuls 1:}  Wie bist du an den Text herangegangen? Hast du eher gedacht \glqq{}Das schaffe ich sowieso nicht\grqq{} oder \glqq{}Ich probiere es einfach mal\grqq{}? Gab es Momente, in denen du dich ertappt hast, wie ein Fixed Mindset hochkam?

\textbf{Impuls 2:} Wie hast du Hindernisse erlebt --- und wie bist du damit umgegangen? Was hast du getan, wenn du etwas nicht sofort verstanden hast? Welche Strategien haben dir geholfen, dranzubleiben? 

\textbf{Impuls 3:} Wie kannst du ein Growth Mindset beim Lesen entwickeln? Was würdest du dir beim nächsten englischen Text selbst sagen, um motiviert zu bleiben?

\section*{Phase~2: Austausch in eurer Lerngruppe}

\textbf{Impuls 4:} Sprecht darüber, ob und wann ihr das Gefühl hattet, an Grenzen zu stoßen. Tauscht Tipps aus, wie man ein Growth Mindset beim Lesen und Verstehen schwieriger Texte fördern kann. Wie könnt ihr euch gegenseitig dabei unterstützen, die Herausforderung als Lernchance zu sehen?

Denkt dran: Auch der Umgang mit wissenschaftlichen Texten auf Englisch ist eine Fähigkeit, die man lernen und durch Übung verbessern kann. Jede Herausforderung macht euch stärker --- genau das ist Growth Mindset! 

\vspace*{10mm}
\printlicense

\printsocials


\end{document}
