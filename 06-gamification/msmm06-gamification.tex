\documentclass{cssheet}

%--------------------------------------------------------------------------------------------------------------
% Basic meta data
%--------------------------------------------------------------------------------------------------------------

\title{Gamification}
\author{Prof. Dr. Christian Spannagel}
\date{\today}
\hypersetup{%
    pdfauthor={\theauthor},%
    pdftitle={\thetitle},%
    pdfsubject={Aufgaben im Kontext der Veranstaltung Mehr Spaß mit Mathe!},%
    pdfkeywords={algebra}
}

%--------------------------------------------------------------------------------------------------------------
% document
%--------------------------------------------------------------------------------------------------------------

\begin{document}
\printtitle

Die Basis ist der Text von Schedler, M. (2020). Mit Gamification spielend die Schulen verändern. Gamification als Zaubermittel für motivierendes Lernen? \emph{F\&E Edition, 26}, 25--37.

\section*{Phase~1: Individuelle Bearbeitung}

\begin{aufgabe}[Deine Erfahrung mit Gamification]
In welchen Situationen in deinem Leben bist zu Gamification begegnet? Wie hat dies auf dich gewirkt?
\end{aufgabe}


\section*{Phase~2: Austausch in eurer Lerngruppe}

\begin{aufgabe}[Gemeinsame Diskussion]
Tauscht euch über Gamification aus:
\begin{itemize}
\item Vergleicht eure Erfahrungen mit Gamification!
\item Diskutiert: Wie verhält sich Gamification zu den bisher im Rahmen der Lehrveranstaltung besprochenen Themen?
\end{itemize}
\end{aufgabe}

\vspace*{5cm}
\printlicense

\printsocials


\end{document}
