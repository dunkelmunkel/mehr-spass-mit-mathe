\documentclass{../cssheet}

%--------------------------------------------------------------------------------------------------------------
% Basic meta data
%--------------------------------------------------------------------------------------------------------------

\title{Selbstwirksamkeitserwartung}
\author{Prof. Dr. Christian Spannagel}
\date{\today}
\hypersetup{%
    pdfauthor={\theauthor},%
    pdftitle={\thetitle},%
    pdfsubject={Aufgaben im Kontext der Veranstaltung Mehr Spaß mit Mathe!},%
    pdfkeywords={algebra}
}

%--------------------------------------------------------------------------------------------------------------
% document
%--------------------------------------------------------------------------------------------------------------

\begin{document}
\printtitle

\textbf{Vorbemerkung:}  Auch bei dieser Aufgabe gibt es wieder zwei Phasen: In Phase~1 reflektiert ihr eure eigene Selbstwirksamkeiterwartung in verschiedenen Bereichen. In Phase~2 trefft ihr euch in eurer Lerngruppe und tauscht eure Gedanken aus. Haltet eure Ergebnisse übersichtlich fest, zum Beispiel in einer Mindmap!

Die Basis aller Überlegungen ist der Text von Schwarzer und Jerusalem (2002).\footnote{Schwarzer, R. \& Jerusalem, M. (2002). Das Konzept der Selbstwirksamkeit. In M.  \emph{Zeitschrift für Pädagogik, 44}, Beiheft, 28--53.}

\section*{Phase~1: Reflexion}

\textbf{Aufgabe 1 (Selbstwirksamkeit):}  Selbstwirksamkeitserwartung (neben einer generellen Form) in der Regel inhaltsspezifisch. In welchen Bereichen hast du eine hohe Selbstwirksamkeitserwartung und warum? In welchen Bereichen hast du eine niedrige Selbstwirksamkeitserwartung und warum? 

\emph{Hinweis: Bei dieser Aufgabe geht es nicht unbedingt um Mathematik. Suche dir andere Bereiche, in denen du eine hohe bzw. niedrige Selbstwirksamkeitserwartung hast.}

\textbf{Aufgabe 2 (Veränderungen):} Betrachte nun die Bereiche aus Aufgabe~1, in denen du eine niedrige Selbstwirksamkeitserwartung hast. Was könnte dir dabei helfen, eine höhere Selbwirksamkeitserwartung zu entwickeln? Denke dabei auch die vier Quellen von Selbstwirksamkeitserwartung aus dem Artikel.


\section*{Phase~2: Austausch in eurer Lerngruppe}

\textbf{Aufgabe 3 (Gemeinsamkeiten und Unterschiede):} Tauscht euch in eurer Lerngruppe über eure persönlichen Reflexionen aus. Könnt ihr Gemeinsamkeiten finden? Worin unterscheiden sich eure Wahrnehmungen? Welche neuen Perspektiven ergeben sich durch eure Diskussion?


\vspace*{10mm}
\printlicense

\printsocials


\end{document}
