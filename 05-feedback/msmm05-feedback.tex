\documentclass{cssheet}

%--------------------------------------------------------------------------------------------------------------
% Basic meta data
%--------------------------------------------------------------------------------------------------------------

\title{Feedback}
\author{Prof. Dr. Christian Spannagel}
\date{\today}
\hypersetup{%
    pdfauthor={\theauthor},%
    pdftitle={\thetitle},%
    pdfsubject={Aufgaben im Kontext der Veranstaltung Mehr Spaß mit Mathe!},%
    pdfkeywords={algebra}
}

%--------------------------------------------------------------------------------------------------------------
% document
%--------------------------------------------------------------------------------------------------------------

\begin{document}
\printtitle

Die Basis ist der Text von Hattie, J., \& Timperley, H. (2007). The power of feedback. \emph{Review of educational research, 77}(1), 81-112.

\section*{Phase~1: Individuelle Bearbeitung}

\begin{aufgabe}[Funktionen von Feedback]
Lies den Text aufmerksam und fasse die drei zentralen Fragen in eigenen Worten zusammen, die laut Hattie und Timperley durch effektives Feedback beantwortet werden sollen.
\end{aufgabe}

\begin{aufgabe}[Eigene Erfahrungen mit Feedback]
Denke an Situationen aus deiner Schulzeit oder deinem Studium, in denen du Feedback im Mathematikunterricht erhalten hast.
\begin{itemize}
\item Welche Art von Feedback hast du bekommen – und von wem?
\item Welche dieser Rückmeldungen haben dir wirklich weitergeholfen – und warum?
\item Gab es Feedback, das demotivierend oder unklar war? Was hat gefehlt?
\end{itemize}
\end{aufgabe}

\begin{aufgabe}[Feedback-Ebenen]
Hattie und Timperley unterscheiden verschiedene Ebenen von Feedback (Feedback zur Aufgabe, zum Prozess, zur Selbstregulation und zur Person).
\begin{itemize}
\item Welche dieser Ebenen erleb(te)st du im Mathematikunterricht besonders häufig --- als Lernender oder Lehrender?
\item Welche Ebene hältst du für besonders wirksam --- und weshalb?
\end{itemize}
\end{aufgabe}

\section*{Phase~2: Austausch in eurer Lerngruppe}

\begin{aufgabe}[Gemeinsame Diskussion]
Tauscht euch über eure Reflexionen aus:
\begin{itemize}
\item Welche Rolle spielt Feedback in euren bisherigen Mathematik-Erfahrungen?
\item In welchen Kontexten habt ihr besonders hilfreiches Feedback erlebt?
\item Welche Gemeinsamkeiten und Unterschiede gibt es zwischen euren Erfahrungen?
\item Welche Impulse aus dem Text erscheinen euch besonders praxisrelevant?
\end{itemize}
\end{aufgabe}

\newpage
\printlicense

\printsocials


\end{document}
